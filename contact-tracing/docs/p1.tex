\documentclass[table, 12pt]{article}
\usepackage{graphicx}
\usepackage[T1]{fontenc}
\usepackage{tocloft}
\usepackage{todonotes}
\usepackage{caption}
\usepackage{hyperref}
\usepackage{booktabs}
\usepackage{listings}
\usepackage{pdfpages}
\usepackage{pdflscape}
\usepackage{textpos}
\usepackage{scrhack}
\usepackage{xcolor}
\usepackage{float}
\usepackage{longtable}
\usepackage{enumitem}
\usepackage{tasks}
\usepackage{tabularx}
\usepackage{titlesec}
\usepackage{listing}
\usepackage{graphicx}


\titleformat{\paragraph}
{\normalfont\normalsize\bfseries}{\theparagraph}{1em}{}
\titlespacing*{\paragraph}
{0pt}{3.25ex plus 1ex minus .2ex}{1.5ex plus .2ex}

\begin{document}
\begin{titlepage}
    \centering
    {\scshape\large AY 2020/2021 \par}
    \vfill
    \includegraphics[width=100pt]{assets/logo-polimi-new}\par\vspace{1cm}
    {\scshape\LARGE Politecnico di Milano \par}
    \vspace{1.5cm}
    {\huge\bfseries Middleware Technologies\\Contact Tracing with IoT Devices\par}
    \vspace{2cm}
    {\Large {Federico Armellini\quad Luca Pirovano\quad Nicolò Sonnino}\par}
    \vfill
    {\large Professor\par
        Luca \textsc{Mottola}}
    \vfill
    {\large \textbf{Version 1.0}\\ \today \par}
\end{titlepage}
\hypersetup{%
    pdfborder = {0 0 0}
}
\thispagestyle{plain}
\pagenumbering{gobble}
\mbox{}
\newpage
\pagenumbering{roman}
\tableofcontents
\newpage
\pagenumbering{arabic}
\section{Introduction}
\subsection{Description of the project}
People roaming in a given location carry IoT devices.

The devices use the radio as a proximity sensor. Every time
two such devices are within the same broadcast domain, that is, at 1-hop distance from each other, the two
people wearing the devices are considered to be in contact. 

The contacts between people’s devices are
periodically reported to the backend on the regular Internet. Whenever one device signals an event of interest,
every other device that was in contact with the former must be informed.


\section{Solution Overview}


\subsection{Assumptions}
\begin{enumerate}[label=\textbf{A\arabic*:}]
    \item The IoT devices may be assumed to be constantly reachable, possibly across multiple hops, from a single static IoT device that acts as a IPv6 border router, that is, you don’t need to consider cases of network partitions.
    \item The IoT part may be developed and tested entirely using the COOJA simulator. To simulate mobility, you may simply move around nodes manually.
    \item Notification at a target device may be accomplished in simple ways, for example, by turning on a LED or printing something out on the serial console.
\end{enumerate}


\end{document}