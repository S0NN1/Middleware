\documentclass[table, 12pt]{article}
\usepackage{graphicx}


\begin{document}

\begin{titlepage}
		\centering
		{\scshape\large AY 2020/2021 \par}
		\vfill
		\includegraphics[width=100pt]{assets/logo-polimi-new}\par\vspace{1cm}
		{\scshape\LARGE Politecnico di Milano \par}
		\vspace{1.5cm}
		{\huge\bfseries Middleware Technologies\\Analysis of COVID-19 Data\par}
		\vspace{2cm}
		{\Large {Federico Armellini\quad Luca Pirovano\quad Nicolò Sonnino}\par}
		\vfill
		{\large Professor\par
			Alessandro \textsc{Margara}}
		\vfill
		{\large \textbf{Version 1.0}\\ \today \par}
	\end{titlepage}

	\thispagestyle{plain}
	\pagenumbering{gobble}
	\mbox{}
	\newpage
	\pagenumbering{roman}
	\tableofcontents
	\newpage
	\pagenumbering{arabic}
	
	
\section{Introduction and assignment}
\subsection{Description of the project}
In this project, you have to implement a program that analyzes open datasets to study the evolution of the
COVID-19 situation worldwide. The program starts from the dataset of new reported cases for each country
daily and computes the following queries:

\begin{enumerate}


\item Seven days moving average of new reported cases, for each county and for each day
\item Percentage increase (with respect to the day before) of the seven days moving average, for each country
and for each day
\item Top 10 countries with the highest percentage increase of the seven days moving average, for each day
\end{enumerate}
You can either use real open datasets 1 or synthetic data generated with the simulator developed for Project 4.
A performance analysis of the proposed solution is appreciated (but not mandatory). In particular, we are
interested in studies that evaluate (1) how the execution time changes when increasing the size of the dataset
and/or number of countries; (2) how the execution time decreases when adding more processing cores/hosts.

\subsection{Assumptions and guidelines}
 \begin{itemize}
 

 
 \item
When using a real dataset, for countries that provide weekly reports, you can assume that the weekly increment is evenly spread across the day of the week.

 \end{itemize}



\section{Solution Overview}
\subsection{Architecture chosen}
Apache Spark



\subsection{Assumptions and Definitions}
\begin{itemize}


\setlength\itemsep{-0.5em}
	\item The dataset is in the csv format. Each line contains: day, rank, number of infected people, number of sane people, number of new infected people (than the day before), the number of new sane people (than the day before).
\end{itemize}

\subsection{General solution}
\subsubsection{Main and Calculate}

\begin{itemize}
\setlength\itemsep{-0.5em}
\item Doing a query that gets how many days and how many countries are in the data.
\item Doing a for loop, we calculate for each country the necessary values for the 3 queries.
\item Printing the results
\end{itemize}

\subsubsection{calculate (single country)}

We handle diffently the first 7 days to the other ones (because they are special, the first doesn't have, for example, a previous moving avarage to compare to).

So, for each day:

\begin{itemize}
\setlength\itemsep{-0.5em}
\item Query the data to get the number of new reported cases
\item Calculating the neeed values for the 3 queries and storing them.
\end{itemize}

\subsection{Data structures}
We use 2 main data structures: DayCountryInfo and Top10Countries.

Then we have highscore (defined as HashMap<Integer, Top10Countries> indexed by day) and query1and2Result (defined as HashMap<Integer, HashMap<Integer, DayCountryInfo>>, indexed by day and country rank) to store the overall result of the 3 queries.

\subsubsection{DayCountryInfo}
DayCountryInfo stores information about a certain country in a certain day (its movingAverageValue, its movingAverageIncrease);

\subsubsection{Top10Countries}
Top10Countries stores in an arraylist, for each day, the information about the top 10 countries, as for query 3.


\subsection{Performance evaluation}

\subsubsection{Variables and parameters}
\begin{flushleft}
\begin{itemize}
\setlength\itemsep{-0.5em}
\item $ queryWhereOnRank = maxContries $
\item $ arrayCopyDays = maxDays/2 $
\item $ queryWhereOnDay = maxDays $
\item $ query3UpdateResults = log2(10) + 10 $
\item $ query1and2UpdateResults = 1 $
\end{itemize}
\end{flushleft}

\subsubsection{Equations}
\begin{flushleft}
\begin{itemize}
\setlength\itemsep{-0.5em}
\item $ maxCountries * ( queryWhereOnRank + maxDays * (arrayCopyDays + queryWhereOnDay + query3UpdateResults + query1and2UpdateResults) )  $
\item $ maxCountries * ( maxContries + maxDays * (maxDays/2 + maxDays + (log2(10)+10) + 1) ) $
\item $ maxCountries * ( maxContries + maxDays * (maxDays* 3/2 + 14.32) ) $
\item $ maxCountries * ( maxContries + maxDays^2 * 3/2 + 14.32 * maxDays) ) $
\item $  maxContries^2 + maxCountries * maxDays^2 * 1.5 + 14.32 * maxDays * maxCountries) $
\end{itemize}
\end{flushleft}


\subsubsection{Conclusion}

Overall performance is (simplified): \\
  $  maxContries^2 + maxCountries * maxDays^2 * 1.5 $

\end{document}